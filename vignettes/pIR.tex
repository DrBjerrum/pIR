%\VignetteIndexEntry{pIR}
%\VignettePackage{pIR}
%\VignetteEngine{utils::Sweave}

\documentclass{article}

\RequirePackage{/Users/yperez/Library/R/3.1/library/BiocStyle/sty/Bioconductor}

\AtBeginDocument{\bibliographystyle{/Users/yperez/Library/R/3.1/library/BiocStyle/sty/unsrturl}}
\newcommand{\exitem}[3]{%
  \item \texttt{\textbackslash#1\{#2\}} #3 \csname#1\endcsname{#2}.%
}

\title{pIR: An \R{} package for isoelectric point prediction based on amino acid sequences}
\author{Yasset Perez-Riverol}

\usepackage{Sweave}
\begin{document}
\Sconcordance{concordance:pIR.tex:/Users/yperez/IdeaProjects/github-repo/R-packages/pIR/vignettes/pIR.Rnw:%
1 5 1 1 2 1 0 1 2 8 1 1 0 219 1 1 2 20 0 1 2 3 1}


\maketitle


%% Abstract and keywords %%%%%%%%%%%%%%%%%%%%%%%%%%%%%%%%
\vskip 0.3in minus 0.1in
\hrule
\begin{abstract}
   Accurate estimation of the isoelectric point value (pI) based on the amino acid sequence becomes critical to perform proteomics experiments. Also, it is one of the most useful electrostatic properties to study peptides and proteins. Different methods has been proposed to compute the theoretical isoelectric point of peptides and proteins using several pK sets \cite{perez2012, cargile2008, bjellqvist1993}. This vignette provides a brief overview of the available interface and functionality as well as a short use case.
\end{abstract}

\textit{Keywords}: proteomics, peptides, proteins, electrophoresis, mass spectrometry, isoelectric point, tutorial.
\vskip 0.1in minus 0.05in
\hrule
\vskip 0.2in minus 0.1in
\vspace{10mm}
%%%%%%%%%%%%%%%%%%%%%%%%%%%%%%%%%%%%%%%%%%%%%%%%%%%%%%%%%%


\tableofcontents

\section{Authoring Sweave / \LaTeX{} package vignettes}

To use with Sweave, add the following to your package \file{DESCRIPTION} file:
\begin{verbatim}
    Suggests: BiocStyle
\end{verbatim}
and add this code chunk to the preamble (between the
\verb+\documentclass{article}+ and \verb+\begin{document}+ latex
commands) of your \texttt{.Rnw} file:
\begin{verbatim}
    <<style-Sweave, eval=TRUE, echo=FALSE, results=tex>>=
    BiocStyle::latex()
    @
\end{verbatim}

To use with \CRANpkg{knitr}, add the following to the \file{DESCRIPTION} file:
\begin{verbatim}
    VignetteBuilder: knitr
    Suggests: BiocStyle, knitr
\end{verbatim}
this to the top of the \texttt{.Rnw} file:
\begin{verbatim}
    %\VignetteEngine{knitr::knitr}
\end{verbatim}
and this to the preamble:
\begin{verbatim}
    <<style-knitr, eval=TRUE, echo=FALSE, results="asis">>=
    BiocStyle::latex()
    @
\end{verbatim}
See \Rcode{?latex} for additional options. \Biocpkg{BiocStyle}
automatically attaches the following \LaTeX{} packages:
\texttt{color}, \texttt{enumitem}, \texttt{fancyhdr},
\texttt{geometry}, \texttt{hyperref}, \texttt{parskip}, \texttt{sectsty}.

Provided the package has been installed, a convenient way to view the
vignette as it is being written is with the command
\begin{verbatim}
    R CMD Sweave --pdf vignette.Rnw
\end{verbatim}
A short-cut useful for checking the \LaTeX{} portion of vignettes is
to toggle evaluation of code chunks to \Rcode{FALSE}
\begin{verbatim}
    SWEAVE_OPTIONS="eval=FALSE" R CMD Sweave --pdf vignette.Rnw
\end{verbatim}
When using \CRANpkg{knitr}, the command to process the vignette is
\begin{verbatim}
    R CMD Sweave --engine=knitr::knitr --pdf vignette.Rnw
\end{verbatim}
By default, \CRANpkg{knitr} automatically caches results of vignette
chunks, greatly accelerating the turnaround time required for
edits. Both the default and \CRANpkg{knitr} incantations create PDF
files using \software{texi2dvi --pdf}; many versions of this software
incorrectly display non-breaking spaces as a tilde, \verb|~|. This can be
remedied (as on the \Bioconductor{} build system) with a final run of
\begin{verbatim}
    R CMD texi2dvi --pdf vignette.tex
    R CMD pdflatex vignette.tex
\end{verbatim}

\section{Style macros}
\Biocpkg{BiocStyle} introduces the following additional markup styling commands
useful in typical \Bioconductor{} vignettes.\\\\
%%
Software:
\begin{itemize}
  \item \verb+\R{}+ and \verb+\Bioconductor{}+ to reference \R{} software and
  the \Bioconductor{} project.
  \exitem{software}{GATK}{to reference third-party software, e.g.,}
\end{itemize}
%\vspace{1em}
%%
Packages:
\begin{itemize}
  \exitem{Biocpkg}{IRanges}{for \Bioconductor{} software packages, including a
  link to the release version landing page,}
  \exitem{Biocannopkg}{org.Hs.eg.db}{for \Bioconductor{} annotation packages,
  including a link to the release version landing page,}
  \exitem{Biocexptpkg}{parathyroidSE}{for \Bioconductor{} experiment data
  packages, including a link to the release version landing page,}
  \exitem{CRANpkg}{data.table}{for \R{} packages available on CRAN, including
  a link to the FHCRC CRAN mirror landing page,}
  \exitem{Githubpkg}{rstudio/rmarkdown}{for \R{} packages available on GitHub, including a link to the package repository,}
  \exitem{Rpackage}{MyPkg}{for \R{} packages that are \emph{not} available on
  \Bioconductor{} or CRAN,}
\end{itemize}
%\vspace{1em}
%%
Code:
\begin{itemize}
  \exitem{Rfunction}{findOverlaps}{for functions}
  \exitem{Robject}{olaps}{for variables}
  \exitem{Rclass}{GRanges}{when referring to a formal class}
  \exitem{Rcode}{log(x)}{for \R{} code,}
\end{itemize}
%\vspace{1em}
%%
Communication:
\begin{itemize}
  \exitem{bioccomment}{additional information for the user}{communicates}
  \exitem{warning}{common pitfalls}{signals}
  \exitem{fixme}{incomplete functionality}{provides an indication of}
\end{itemize}
%\vspace{1em}
%%
General:
\begin{itemize}
  \exitem{email}{user@domain.com}{to provide a linked email address,}
  \exitem{file}{script.R}{for file names and file paths}
\end{itemize}

%---------------------------------------------------------
\section{Title, running headers, and table of contents}
%---------------------------------------------------------

Create a title and running headers by defining the \verb+\bioctitle+
and \verb+\author+ commands in the preamble
\begin{verbatim}
    \bioctitle[Short title for headers]{Full title for title page}
    %% also: \bioctitle{Title used for both header and title page}
    %% or... \title{Title used for both header and title page}
    \author{Iman Author\footnote{iman@author.org}}
\end{verbatim}
Use \verb+\maketitle+ at the start of the document to create the title
in the document.

Use \verb+\tableofcontents+ for a hyperlinked table of contents,
\verb+\section+, \verb+\subsection+, \verb+\subsubsection+ for
structuring your vignette.

Formatting of subsections and subsubsections are as follows.

\subsection{This is a subsection}

\subsubsection{This is a subsubsection}

%---------------------------------------------------------
\section{Figures}\label{incfig}
%---------------------------------------------------------

Besides the usual \LaTeX{} capabilities (\verb+figure+ environment and
\verb+\includegraphics+ command), \file{Bioconductor.sty} defines a
macro \verb+\incfig[placement]{filename}{width}{shorttitle}{extendedcaption}+,
which expects four arguments:
\begin{description}%[font=\texttt]
\item[filename] The name of the figure file, also used as the label by
  which the float can be referred to by \verb+\ref{}+. Some
  \software{Sweave} and \CRANpkg{knitr} options place figures in a
  subdirectory; unless \Rcode{short.fignames=TRUE} is set the full file name,
  including the subdirectory and any prefixes, should be provided. By default,
  these are \file{<sweavename>-} for \Rpackage{Sweave} and \file{figure/} for
  \CRANpkg{knitr}. Please note the different naming scheme used by
  \Rpackage{knitr}: figure files are named \file{<chunkname>-i} where \emph{i}
  is the number of the plot generated in the chunk.
\item[width] Figure width.
\item[shorttitle] A short description, used in the list of figures and
  printed in bold as the first part of the caption.
\item[extendedcaption] Continuation of the figure caption.
\end{description}
The optional \textbf{placement} specifier controls where the figure is placed
on page and takes the usual values allowed by \LaTeX{} floats, i.e., a list
containing \verb+t+, \verb+b+, \verb+p+, or \verb+h+, where letters enumerate
permitted placements. If no placement specifier is given, the default
\verb+tbp+ is assumed.

For \verb+incfig+ with Sweave, use
\begin{verbatim}
    <<figureexample, fig=TRUE, include=FALSE, width=4.2, height=4.6>>=
    v = seq(0, 60i, length=1000)
    plot(abs(v)*exp(v), type="l", col="Royalblue")
    @
    \incfig{LatexStyle-figureexample}{0.25\textwidth}{A curve.}
      {The code that creates this figure is shown in the code chunk.}
    as shown in Figure~\ref{LatexStyle-figureexample}.
\end{verbatim}
This results in
\begin{Schunk}
\begin{Sinput}
> v = seq(0, 60i, length=1000)
> plot(abs(v)*exp(v), type="l", col="Royalblue")
\end{Sinput}
\end{Schunk}

\incfig{LatexStyle-figureexample}{0.25\textwidth}{A curve.}
  {The code that creates this figure is shown in the code chunk.}
as shown in Figure~\ref{LatexStyle-figureexample}.  When the option
\Rcode{short.fignames} is set to \Rcode{TRUE}, figure names used by
\verb+\incfig+ and \verb+\ref+ do not contain any prefix and are
identical to the corresponding code chunk labels (plus figure number in case of
\Rpackage{knitr}). For example, in Sweave the respective code for the above
example would be \verb+\incfig{figureexample}{...}{...}{...}+ and
\verb+\ref{figureexample}+, while in \Rpackage{knitr} these are expected to be
\verb+\incfig{figureexample-1}{...}{...}{...}+ and
\verb+\ref{figureexample-1}+.

For \verb+\incfig+ with \Rpackage{knitr}, use the option
\Rcode{fig.show='hide'} rather than \Rcode{include=FALSE}. The
\Rpackage{knitr}-equivalent code for
Figure~\ref{LatexStyle-figureexample} is:
\begin{verbatim}
    <<figureexample, fig.show='hide', fig.width=4.2, fig.height=4.6>>=
    v = seq(0, 60i, length=1000)
    plot(abs(v)*exp(v), type="l", col="Royalblue")
    @
\end{verbatim}
Note the difference in option names setting the figure width and
height compared to \Rpackage{Sweave}.  Unless
\Rcode{short.fignames=TRUE} is set, use the default \file{figure/}
prefix when inserting and referring to figures, e.g.:
\begin{verbatim}
    \incfig{figure/figureexample-1}{0.25\textwidth}{A curve.}
      {The code that creates this figure is shown in the code chunk.}
\end{verbatim}
A custom prefix for figure file names can be passed to
\Rfunction{latex} using the \Rcode{fig.path} option. When
\Rcode{short.fignames=TRUE}, figures can be referred to directly by
code chunk labels, as described earlier in this section.


%---------------------------------------------------------
\section{Session info}
%---------------------------------------------------------

Here is the output of \Rfunction{sessionInfo} on the system on which
this document was compiled:
\begin{Schunk}
\begin{Sinput}
> toLatex(sessionInfo())
\end{Sinput}
\begin{itemize}\raggedright
  \item R version 3.1.3 (2015-03-09), \verb|x86_64-apple-darwin10.8.0|
  \item Locale: \verb|en_GB.UTF-8/en_GB.UTF-8/en_GB.UTF-8/C/en_GB.UTF-8/en_GB.UTF-8|
  \item Base packages: base, datasets, graphics, grDevices, methods, stats,
    utils
  \item Loaded via a namespace (and not attached): BiocStyle~1.2.0, tools~3.1.3
\end{itemize}\end{Schunk}

\bibliography{Bioc}

\end{document}
